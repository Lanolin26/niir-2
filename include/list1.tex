\section{Выполнение задания}

\subsection{Этап 2. Изучение систем виртуализации и контейнеризации}
В ходе исследования были найдены и описаны основные понятия понятий и технологий виртуализации и контейнеризации.
Они включали в себя различные типы виртуализации, таких как полная виртуализация, 
паравиртуализация и контейнеризация, а также обзор инструментов и технологий, таких как Docker, 
Kubernetes, VMware и Proxmox (приложение \hyperlink{app-a}{А}). 

\subsection{Этап 3. Описание работы, развертывания и администрирования Proxmox}
В ходе работы было проведено описание работы, развертывания и администрирования Proxmox (приложение \hyperlink{app-a}{А}).
Основываясь на официальной документации Proxmox, был рассмотрен дистрибутив и его возможности. Так же, по
документации, а так же опыта работы с виртуальными машинами была описана работа с виртуальными машинами. 
Создание и настройка виртуальной машины была рассмотрена на примере установки программного обеспечения Nextcloud.

\subsection{Этап 4. Описание работы, развертывания и администрирования K8S}
Описана работа с системой контейнеризации K8S, а также развертывание и администрирование контейнеров (приложение \hyperlink{app-a}{А}).
Основываясь на официальной документации, было рассмотрено программное обеспечение K8S, его состав, а так же
базовые возможности управления приложением внутри K8S.
Работа с K8S по развертыванию приложения и базовое администрирование было рассмотрено на примере установки программного обеспечения Nextcloud, 
используя подготовленный docker образ.

\subsection{Этап 5. Оформление разработанного учебного практикума}
В ходе оформления учебного практикума, был изучен текущий учебный практикум по дисциплине <<Технологии виртуализации>>.
Так же были проведены консультации с научным руководителем.
По итогам был переработан и составлен учебный практикум для изучения систем виртуализации и контейнеризации (приложение \hyperlink{app-b}{Б}).

\subsection{Этап 6. Оформление отчетной документации}
По итогам работы был написан отчет, оформленный в соответствии с методическим пособием \cite{markina}.

\clearpage
