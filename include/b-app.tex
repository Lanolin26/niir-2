\appsection{Приложение Б} \hypertarget{app-b}{\label{app-b}}

\textbf{Учебный практикум}

\annsection{Практика: }{Знакомство с Proxmox. Управление виртуальными машинами}
Каждую часть и этап необходимо подтверждать с помощью скриншотов или записанного видео.

\annsubsection{Часть 1.}{Установка Proxmox}
Цель: познакомиться с установкой Proxmox.

Задание:
\begin{enumerate}[]
    \item Скачать с официального сайта дистрибутив
    \item Произвести установку на виртуальную машину.
    \item Предоставить скриншоты о процессе установки
    \item Войти в систему Proxmox через другим адресным пространством, что и основной интерфейс.
    \item Настроить NAT для этого интерфейса
\end{enumerate}

\annsubsection{Часть 2.}{Создание виртуальной машины}

Цель: научиться создавать виртуальные машины в Proxmox

Задание:
\begin{enumerate}
    \item Скачать дистрибутив Debian 11.
    \item Загрузить дистрибутив в локальное хранилище Proxmox
    \item Создать виртуальную машину:
    \begin{itemize}
        \item Название: фамилия\_имя в латинице
        \item Disk size (GiB): 16
        \item Sockets: 1
        \item Cores: 1
        \item Memory (MiB): 4096
        \item Brige: vmbr0
    \end{itemize}
    \item Произвести запуск созданной виртуальной машины.
    \item С помощью консоли Proxmox произвести установку ОС
    \item Вручную сделать статичный IP адрес в ОС
    \item Проверить работу сети и доступа в интернет
\end{enumerate}

\annsubsection{Часть 3.}{Подключение к виртуальной машине.}

Цель: настроить на виртуальной машине удаленное подключение по протоколу SSH

Задание:
\begin{enumerate}
    \item Войти в консоль виртуальной машины Proxmox
    \item Войти в ОС из-под root пользователя
    \item Произвести установку пакета ssh
    \item Из хост-системы произвести подключение к виртуальной машине
\end{enumerate}

\annsubsection{Часть 4.}{Установка Nextcloud}

Цель: установить Nextcloud

Задание:
\begin{enumerate}
    \item Скачать архив с дистрибутивом Nextcloud на виртуальную машину
    \item Установить необходимые пакеты для LAMP стека и для Nextcloud 
    \item В СУБД MySQL/MariaDB создать пользователя с паролем и отдельную БД.
    \item Для Apache произвести настройку для Nextcloud
    \item Включить сайт apache
    \item Войти на сайт Nextcloud: http://<ip-address>[:<port>]/
    \item Следовать установщику. В случае ошибок - исправлять и заново производить установку
    \item После установки:
    \begin{enumerate} 
        \item Создать пользователя
        \item Войти под ним
        \item Загрузить любой файл
        \item Убедиться, что файл загрузился
        \item Скачать файл обратно
    \end{enumerate}
\end{enumerate}

\annsection{Практика: }{Знакомство с Docker. Управление контейнерами}

\annsubsection{Часть 1.}{Установка Docker}

Цель: установить docker на целевой или гостевой ОС

Задание:
\begin{enumerate}
    \item Установить Docker штатными средствами: https://docs.docker.com/engine/install/
    \item Проверить работу Docker при помощи команды: docker –version
    \item Проверить корректность запуска docker image: docker run hello-world
\end{enumerate}

\annsubsection{Часть 2.}{Написание Dockerfile}

Цель: Изучить возможности написания Docker image при помощи Dockerfile

Задание:
\begin{enumerate}
    \item Изучить Reference manual по Dockerfile
    \item Написать простейший image: 
    \begin{itemize}
        \item выводить страницу hello.html
        \item в hello.html находится ваше ФИО и группа
    \end{itemize}
    \item Файл hello.html должен находится в volume
    \item Собрать образ
    \item Запустить образ с пробросом порта
    \item Проверить работу образа: curl http://localhost:<port>/hello.html 
\end{enumerate}

\annsubsection{Часть 3.}{Параметры запуска образов}

Цель: Изучить команды docker

Задание:
\begin{enumerate}
    \item Образ, собранный в предыдущей части запустить на другом порту. Проверить работу
    \item Тот же образ запустить с другим содержимым volume
    \item Пересобрать образ - убрать volume, файл hello.html копировать в слой образа
    \item Запушить образ в hub.docker.com
    \item Добавить к собранному образу новый тег.
\end{enumerate}

\annsection{Практика:}{Знакомство с K8S. Развертывание облачных сервисов}

\annsubsection{Часть 1.}{Установка K8S}

Цель: Установить на целевую ОС K8S

Задание:
\begin{enumerate}
    \item Установить K8S
    \item Для Windows: можно использовать Docker Desktop с включенной поддержкой Kubernetes
    \item Для Linux: использовать minikube или другие аналоги
    \item Проверить работу K8S. В качестве показателя работоспособности кластера использовать команду kubectl cluster-info
\end{enumerate}

\annsubsection{Часть 2.}{Создание манифеста}

Цель: Изучить правила составления манифестов

Задание:
\begin{enumerate}
    \item Изучить Reference manual по манифестам Kubernetes
    \item Составить манифест Deployments для образа, созданного ранее
\end{enumerate}

\annsubsection{Часть 3.}{Деплой манифеста и проверка работы}

Цель: Научиться использовать консольную утилиту для работы с кластером

Задание:
\begin{enumerate}
    \item Произвести применение манифеста в кластере
    \item Проверить работу запущенного Deployments
    \begin{itemize}
        \item С помощью kubectl describe deployment <название>
        \item С помощью kubectl port-forward --address 0.0.0.0 deployment/<название> <port>:<port> и  curl http://127.0.0.1:<port>/hello.html
    \end{itemize}
\end{enumerate}

\annsubsection{Часть 4.}{Параметры подов в манифесте}

Цель: Изучить другие возможности управления объектами K8S

Задание:
\begin{enumerate}
    \item Для созданного Deployments добавить значение по умолчанию количества реплик = 2
    \item Для созданного Deployments добавить пробу/пробы
\end{enumerate}

\clearpage