\appsection{Приложение А} \hypertarget{app-a}{\label{app-a}}

\textbf{Алгоритмы развертывания и управления облачными сервисами с помощью Proxmox и K8}

\annotation{Введение}
 
В настоящее время довольно существенной частью IT становятся виртуальные машины и средства виртуализации контейнеризации. Выделять для одной программы целый сервер непродуктивно с точки зрения потребление компьютерных ресурсов. На одном сервере можно запустить несколько виртуальных машин или контейнеров с разными приложениями и операционными системами, соблюдая их безопасность и корректное распределение ресурсов сервера.  

В любой IT компании будет присутствовать средство виртуализации и/или контейнеризации. Однако для реализации частных средств необходимо знать и понимать, как настраиваются хостовые операционные системы, создаются и запускаются контейнеры.

\annsection{1}{Изучение систем виртуализации и контейнеризации}

Виртуализация - это процесс переноса физической системы в виртуальную среду. Это означает создание виртуальных версий устройств и ресурсов (от серверов до операционных систем). Виртуализация вычислительных  ресурсов позволяет одному физическому серверу стать узлом в группе виртуальных серверов, использующих одни и те же ресурсы. Эти методы позволяют сократить как количество физических серверов, так и число системных администраторов за счет снижения необходимости в обслуживании и администрировании.

Возможности существующих технологий виртуализации зависит от уровня, на котором они работают. Согласно эталонной модели, виртуализации может работать на уровне системы инструкций (ISA) или на уровне двоичного интерфейса приложений (ABI). Поэтому технологии виртуализации можно разделить на виртуализацию на уровне операционной системы и виртуализацию на уровне процессора (аппаратную виртуализацию).

\annsubsection{1.1}{Виртуализация}

Основными технологиями аппаратной виртуализации являются полная виртуализация и паравиртуализация. В первом случае интерфейс виртуализации, идентичен интерфейсу хост-машины. Паравиртуализация, как следует из названия, предоставляет интерфейс, отличный от интерфейса хост-машины. Существует два типа гипервизоров: первый и второй.

Гипервизоры первого типа - это программное обеспечение, которое работает непосредственно на физическом оборудовании. Архитектурно гипервизоры типа 1 располагаются непосредственно на аппаратном уровне и поэтому могут напрямую использовать аппаратные ресурсы.

Второй тип гипервизора устанавливается поверх базовой операционной системы (например, Linux или Windows), работающей на физическом сервере. В этом случае операционная система хоста работает непосредственно на оборудовании, а сам гипервизор представляет собой программный слой, на котором могут работать различные гостевые операционные системы.

Согласно \cite{huai2007civic}, гипервизоры имеют четыре основные характеристики:

\begin{itemize}
    \item Прозрачность. Это означает, что программное обеспечение может быть запущено непосредственно в среде виртуальной машины.
    \item Изоляция. Гипервизор позволяет размещать несколько экземпляров виртуальных машин на одном физическом сервере. Каждая виртуальная машина может установить свою собственную версию программного обеспечения без необходимости учитывать совместимость с программным обеспечением других виртуальных машин.
    \item Инкапсуляция. Вся система виртуальной машины расположена на виртуальном жестком диске, который является обычным файлом хост-машины.
    \item Управляемость. Программные интерфейсы предназначены для добавления, изменения и удаления виртуального оборудования, а также для выполнения таких операций с виртуальными машинами, как запуск, выключение и гибернация. Виртуальные машины могут полностью управляться через программный интерфейс
\end{itemize}

Несмотря на перечисленные выше преимущества, виртуализация также имеет недостатки, о чем говорится в \cite{daniels2009server}, где выделены следующие из них:

\begin{itemize}
    \item превышение физических ресурсов при выделении памяти – потенциальная проблема в виртуализированной среде; 
    \item перенос существующих приложений в виртуальную среду может быть изначально дорогостоящим; 
    \item если гипервизор выходит из строя, все установленные на нем виртуальные машины могут быть потеряны; 
\end{itemize}

Известными гипервизорами или продуктами виртуализации являются:

\begin{itemize}
    \item KVM \cite{KVM}, модуль ядра для ядра Linux, позволяющий получить доступ к возможностям виртуализации реального оборудования, и эмулятор, обычно используемый с KVM, qemu \cite{qemu}, который эмулирует остальное оборудование (тип 2),
    \item Proxmox \cite{proxmox}, свободный гипервизор, работающий непосредственно на аппаратном обеспечении (тип 1)
    \item Hyper-V \cite{Hyper-V}, гипервизор от Microsoft, интегрированный в различные версии Windows (тип 1)
    \item VMware Workstation \cite{VMware}, проприетарное программное обеспечение для виртуализации от VMware (тип 2)
    \item VirtualBox \cite{VirtualBox}, бесплатное, независимое от платформы решение для виртуализации от Oracle (тип 2)
\end{itemize}

\annsubsection{1.2}{Контейнеризация}

Виртуализация на уровне ОС - это метод виртуализации сервера, при котором ядро операционной системы позволяет существовать нескольким изолированным экземплярам пользовательского пространства вместо единого пространства. Виртуализация контейнеров использует функциональность ядра для создания изолированных сред для процессов. В отличие от виртуализации на базе гипервизора, контейнеры не приобретают собственное виртуальное оборудование, а используют оборудование хост-системы. Поэтому программное обеспечение, работающее в контейнерах, должно напрямую взаимодействовать с ядром хоста и работать на операционной системе и процессорной архитектуре, на которой работает хост.

Примерами контейнерной виртуализации являются LXC, FreeBSD Jail, OpenVZ и Docker.

Согласно \cite{alapati2016modern}, особенности, которые делают контейнеры полезными и привлекательными для разработчиков приложений и менеджеров инфраструктуры, следующие: 

\begin{itemize}
    \item Быстрое развертывание приложений. Из-за легковесности контейнеров их запуск происходит очень быстро.
    \item Масштабируемость. Контейнеры могут работать практически на любой системе Linux и могут быть развернуты в облаке, на настольных компьютерах или на физических серверах. Перемещение контейнеров из среды в облако или на физические серверы происходит быстро и легко. Также легко увеличить количество контейнеров и ресурсов.
    \item Производительность. Контейнеры не требуют отдельной полноценной операционной системы, поэтому на одном физическом сервере может быть запущено вдвое больше контейнеров, чем виртуальных машин \cite{xavier2013performance}. Контейнеры не используют ресурсы сервера (память и процессор) в режиме простоя, в отличие от виртуальных машин, которые изначально занимают эти ресурсы.
    \item Легкость обслуживания. Управлять жизненным циклом виртуальной машины непросто. Это связано с тем, что каждый сервер имеет как минимум две операционные системы - гипервизор, который необходимо исправлять и обновлять, и гостевую операционную систему в виртуальной машине.
    \item Микросервисы. В отличие от контейнеризации, виртуализация требует отдельной виртуальной машины для каждого процесса операционной системы и поэтому не подходит для микросервисов, которые могут использовать сотни тысяч процессов.
    \item Переносимость. Контейнеры переносимы, поэтому приложения можно объединять в единое целое и развертывать в различных средах без каких-либо изменений в самом контейнере.
\end{itemize}

Контейнеры имеют множество преимуществ, но есть и области, в которых виртуализация не может быть заменена контейнеризацией. Контейнеры по сути являются изолированными копиями операционной системы хоста с собственным пользовательским окружением и базовыми ресурсами, поэтому нет возможности глубокой настройки операционной системы, самостоятельной установки или использования операционной системы, которая не является операционной системой хоста.

\annsection{2}{Описание работы, развертывания и администрирования Proxmox}

\annsubsection{2.1}{Описания развертывания и администрирования системы Proxmox}

Proxmox - система виртуализации первого типа. Она основывается на Debian и тесно интегрирует гипервизор KVM и контейнеры Linux (LXC), программно-определяемое хранилище и сетевые функции на одной платформе. Благодаря интегрированному пользовательскому веб-интерфейсу можно с легкостью управлять виртуальными машинами и контейнерами, высокой доступностью кластеров или интегрированными инструментами аварийного восстановления \cite{proxmox}.

Управление виртуальными машинами и администрирование самого сервера производятся через веб-интерфейс либо через стандартный интерфейс командной строки Linux. 

Программное обеспечение Proxmox после установки сразу готово к виртуализации машин на базе Windows Server, Ubuntu, CentOS 8 или других платформ.

Установка Proxmox производится так же, как и установка любой ОС на реальное железо: скачивается образ, сервер загружается с загрузочного образа

\annsubsection{2.2}{Описание работы создание и управление ВМ}

Алгоритм развертывания условно можно разделить на 2 этапа: создание объекта “Виртуальная машина” в гипервизоре и установка и настройка ОС на созданную виртуальную машину.

Этап 1. Создание объекта “Виртуальная машина”:
\begin{enumerate}
    \item Необходимо войти в интерфейс системы виртуализации
    \item Создать объект виртуальная машина
    \item Назначить обязательные ресурсы - CPU, RAM, HDD
    \item Подключить образ устанавливаемой системы.
\end{enumerate}

Этап 2. Установка и настройка ОС на созданную виртуальную машину:
\begin{enumerate}
    \item В интерфейсе системы виртуализации присутствует консоль, которая эмулирует подключение монитора и позволяет управлять виртуальной машиной
    \item После запуска виртуальной машины и подключение к консоли, происходит установка ОС, слабо отличимая от установки ОС на реальное железо.
\end{enumerate}

Автоматизировать процесс развертывания виртуальной машины возможно. Существует множество вариантов, одни из них: 
\begin{itemize}
    \item написание скриптов
    \item написание Ansible Playbook
    \item Terraform + Packer
\end{itemize}

Управление виртуальными машинами происходит через Web-интерфейс. В нем располагаются основные элементы управления, создания, изменения ВМ.

\annsubsection{2.3}{Пример установки Nextcloud}

Для установки Nextcloud в первую очередь понадобится готовый Proxmox.

Следующий этап - создание виртуальной машины и установка на нее гостевой ОС ubuntu/debian.

Далее установка Nextcloud происходит так, как бы производилась установка на реальное железо \cite{nextcloud}.

Используем  .deb пакеты для установки необходимых и рекомендуемых модулей для типичной установки Nextcloud с использованием Apache и MariaDB, выполнив следующие команды в терминале:

\begin{lstlisting}[language=Bash]
    sudo apt update
    sudo apt install apache2 mariadb-server libapache2-mod-php7.4
    sudo apt install php7.4-gd php7.4-mysql php7.4-curl php7.4-mbstring php7.4-intl
    sudo apt install php7.4-gmp php7.4-bcmath php-imagick php7.4-xml php7.4-zip
\end{lstlisting}

Теперь необходимо создать пользователя базы данных и саму базу данных с помощью интерфейса командной строки MySQL. Таблицы базы данных будут созданы Nextcloud при первом входе в систему.

Чтобы запустить режим командной строки MySQL, выполните следующую команду и нажмите клавишу ввода, когда появится запрос на ввод пароля:

\begin{lstlisting}[language=Bash]
sudo /etc/init.d/mysql start
sudo mysql -uroot -p
\end{lstlisting}

После этого появится приглашение MariaDB [root]>. Теперь введите следующие строки, заменив имя пользователя и пароль на соответствующие значения, и подтвердите их клавишей Enter:

\begin{lstlisting}[language=sql]
CREATE USER 'username'@'localhost' IDENTIFIED BY 'password';
CREATE DATABASE IF NOT EXISTS nextcloud CHARACTER SET utf8mb4 COLLATE utf8mb4_general_ci;
GRANT ALL PRIVILEGES ON nextcloud.* TO 'username'@'localhost';
FLUSH PRIVILEGES;
\end{lstlisting}

Вы можете выйти из MariaDB, введя:

\begin{lstlisting}[language=bash]
quit;
\end{lstlisting}

Теперь загрузите архив последней версии Nextcloud: 
\begin{itemize}
    \item Перейдите на страницу загрузки Nextcloud. 
    \item Перейдите в раздел "Загрузить сервер Nextcloud" > "Загрузить" > "Архивный файл для владельцев сервера" и загрузите архив tar.bz2 или .zip. 
    \item Загрузится файл с именем nextcloud-x.y.z.tar.bz2 или nextcloud-x.y.z.zip (где x.y.z - номер версии). 
\end{itemize}

Теперь вы можете извлечь содержимое архива. Запустите соответствующую команду распаковки для вашего типа архива:

\begin{lstlisting}[language=bash]
tar -xjvf nextcloud-x.y.z.tar.bz2
unzip nextcloud-x.y.z.zip
\end{lstlisting}

В результате распаковывается один каталог Nextcloud. Скопируйте каталог Nextcloud в конечный пункт назначения. Если вы используете HTTP-сервер Apache, вы можете безопасно установить Nextcloud в корень документа Apache:

\begin{lstlisting}[language=bash]
cp -r nextcloud /path/to/webserver/document-root
\end{lstlisting}

где \textit{/path/to/webserver/document-root} заменяется корнем документа вашего Web-сервера.

\annsection{3}{Описание работы, развертывания и администрирования K8S}

\annsubsection{3.1}{Описание системы}

Kubernetes — это портативная расширяемая платформа с открытым исходным кодом для управления контейнеризованными рабочими нагрузками и сервисами, которая облегчает как декларативную настройку, так и автоматизацию. У платформы есть большая, быстро растущая экосистема. Сервисы, поддержка и инструменты Kubernetes широко доступны. \cite{Kubernetes}

Система реализует архитектуру «ведущий — ведомый»: выделяется подсистема управления кластером, а часть компонентов управляют индивидуальными, ведомыми узлами (называемых собственно узлами Kubernetes)\cite{WhatisKubernetes}.

Подсистема управления обеспечивает коммуникацию и распределение нагрузки внутри кластера; компоненты подсистемы могут выполняться на одном или на нескольких параллельно работающих ведущих узлах, совместно обеспечивающих режим высокой доступности.

Процедура работы Kubernetes состоит в том, что ресурсы узлов динамически распределяются между выполняемыми на них подами. Каждый узел в кластере содержит ряд типовых компонентов.

\annsubsection{3.2}{Описание работы, создание и управление контейнерами}

Запуск контейнера начинается с создание образа. Можно использовать готовые образы, расположенные на популярных сервисах раздачи образов например \href{https://hub.docker.com/}{hub.docker.com}. 

Образы создаются с использованием технологии Docker.

Развертывание образов в кластер K8S осуществляется с помощью манифестов. Манифест содержит информацию о том, что должно получиться после создание пода, написанная в декларативном стиле, контроль за жизненным циклом контейнера и осуществление мониторинга и балансировки нагрузки.

В официальной документации и описании kubernetes описываются все параметры, которые могут понадобится в манифесте.

Простое развертывание контейнера можно свести к двум этапам: создание образа и развертывание.

Создание образа - опциональный этап, существует большое количество готовых образов для использование. Создание сводится к написанию Dockerfile манифеста, сборки образа и публикация в репозиторий.

Развертывание сводится к написанию манифестов. Можно писать манифесты вручную и запускать их, можно воспользоваться специальными утилитами, например helm, для кастомизации манифестов.

Развертывание манифеста в кластер kubernetes достаточно просто - выполнить команду apply к манифесту, имея активное подключение к кластеру.

\annsubsection{3.3}{Пример установки Nextcloud}

Для начала, необходимо иметь работающий кластер K8S. Можно использовать minikube.

Необходимо создать манифест для работы с контейнерами. В качестве примера воспользуемся: 

\begin{lstlisting}[language=yaml]
---
apiVersion: apps/v1
kind: Deployment
metadata:
  labels:
    io.kompose.service: app
  name: app
spec:
  replicas: 1
  selector:
    matchLabels:
      io.kompose.service: app
  strategy:
    type: Recreate
  template:
    metadata:
        io.kompose.service: app
    spec:
      containers:
        - env:
            - name: MYSQL_DATABASE
              value: nextcloud
            - name: MYSQL_HOST
              value: db
            - name: MYSQL_PASSWORD
              value: '111'
            - name: MYSQL_USER
              value: nextcloud
          image: nextcloud
          name: app
          ports:
            - containerPort: 80
      restartPolicy: Always
--- 
apiVersion: apps/v1
kind: Deployment
metadata:
  name: db
spec:
  replicas: 1
  selector:
    matchLabels:
      io.kompose.service: db
  strategy:
    type: Recreate
  template:
    spec:
      containers:
        - args:
            - --transaction-isolation=READ-COMMITTED
            - --log-bin=binlog
            - --binlog-format=ROW
          env:
            - name: MYSQL_DATABASE
              value: nextcloud
            - name: MYSQL_PASSWORD
              value: '''111'''
            - name: MYSQL_ROOT_PASSWORD
              value: '''1111'''
            - name: MYSQL_USER
              value: nextcloud
          image: mariadb:10.6
          name: db
      restartPolicy: Always
status: {}
\end{lstlisting}

После создания файла манифеста применим его:
\begin{lstlisting}[language=bash]
    kubectl apply -f <file>.yml
\end{lstlisting}

Смотрим не получившиеся контейнеры, дожидаемся их окончательного запуска и проверяем работу или исправляем ошибки.

\annotation{Заключение}
В ходе написание работы, были проанализированы различные подходы к развертыванию серверов - виртуализация и контейнеризация. Составлен алгоритм по развертыванию сервисов в них. Из-за огромного разнообразия приложений и способов установки, был составлен общий алгоритм развертывания.

\begingroup
\renewcommand{\section}[2]{\annotation{СПИСОК ИСПОЛЬЗОВАННЫХ ИСТОЧНИКОВ}}
\bibliography{include/biblist}
\endgroup


\clearpage