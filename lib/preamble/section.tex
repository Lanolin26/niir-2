%---Содержание---------------------------------------------------------%
\usepackage{tocloft}
\renewcommand{\cfttoctitlefont}{\hspace{0.38\textwidth}\MakeTextUppercase}
\renewcommand{\cftsecfont}{\hspace{0pt}}            % Имена секций в содержании не жирным шрифтом
\renewcommand\cftsecleader{\cftdotfill{\cftdotsep}} % Точки для секций в содержании
\renewcommand\cftsecpagefont{\mdseries}             % Номера страниц не жирные
% \cftsetpnumwidth{10pt}
\setcounter{tocdepth}{3}                            % Глубина оглавления, до subsubsection
%----------------------------------------------------------------------%

%---Список иллюстративного материала-----------------------------------%
\renewcommand{\cftloftitlefont}{\hspace{0.17\textwidth}\MakeTextUppercase}
\renewcommand{\cftfigfont}{Рисунок }
\addto\captionsrussian{\renewcommand\listfigurename{Список иллюстративного материала}}
%----------------------------------------------------------------------%

%----Список табличного материала---------------------------------------%
\renewcommand{\cftlottitlefont}{\hspace{0.2\textwidth}\MakeTextUppercase}
\renewcommand{\cfttabfont}{Таблица }
\addto\captionsrussian{\renewcommand\listtablename{Список табличного материала}}
%----------------------------------------------------------------------%

%---Формат подрисуночных надписей--------------------------------------%
\RequirePackage{caption}
\DeclareCaptionLabelSeparator{defffis}{ -- } % Разделитель
\captionsetup[figure]{justification=centering, labelsep=defffis, format=plain} % Подпись рисунка по центру
\captionsetup[table]{justification=raggedright, labelsep=defffis, format=plain, singlelinecheck=false} % Подпись таблицы слева
\addto\captionsrussian{\renewcommand{\figurename}{Рисунок}} % Имя фигуры
\addto\captionsrussian{\renewcommand{\lstlistingname}{Листинг}}
\addto\captionsrussian{\renewcommand{\tablename}{Таблица}}
%----------------------------------------------------------------------%

%---Заголовки секций в оглавлении в верхнем регистре-------------------%
% 
\usepackage{textcase}
\makeatletter
\let\oldcontentsline\contentsline
\def\contentsline#1#2{
    \expandafter\ifx\csname l@#1\endcsname\l@section
        \expandafter\@firstoftwo
    \else
        \expandafter\@secondoftwo
    \fi
    {\oldcontentsline{#1}{\MakeTextUppercase{#2}}}
    {\oldcontentsline{#1}{#2}}
}
\makeatother
%----------------------------------------------------------------------%

%---Оформление заголовков----------------------------------------------%
\usepackage[compact,explicit]{titlesec}
\titleformat{\section}{}{}{12.5mm}{\centering{\thesection\quad\MakeTextUppercase{#1}}\vspace{1.5em}}
\titleformat{\subsection}[block]{\vspace{1em}}{}{12.5mm}{\thesubsection\quad#1\vspace{1em}}
\titleformat{\subsubsection}[block]{\vspace{1em}\normalsize}{}{12.5mm}{\thesubsubsection\quad#1\vspace{1em}}
\titleformat{\paragraph}[block]{\normalsize}{}{12.5mm}{\MakeTextUppercase{#1}}
%----------------------------------------------------------------------%

%----Секции без номеров (введение, заключение...), вместо section*{}---%
\newcommand{\anonsection}[1]{
    \phantomsection % Корректный переход по ссылкам в содержании
	\paragraph{\centerline{{#1}}\vspace{1em}}
    \addcontentsline{toc}{section}{#1}
}
%----------------------------------------------------------------------%

%---Секция для аннотации (она не включается в содержание)--------------%
\newcommand{\annotation}[1]{
    \paragraph{\centerline{{#1}}\vspace{1em}}
}
%----------------------------------------------------------------------%

%---Секция для списка иллюстративного материала------------------------%
\newcommand{\lof}{
    \clearpage
    \phantomsection
    \listoffigures
    \addcontentsline{toc}{section}{\listfigurename}
}
%----------------------------------------------------------------------%

%---Секция для списка табличного материала-----------------------------%
\newcommand{\lot}{
    \clearpage
    \phantomsection
    \listoftables
    \addcontentsline{toc}{section}{\listtablename}
}
%----------------------------------------------------------------------%

%----Секции для приложений---------------------------------------------%
\newcommand{\appsection}[1]{
    \clearpage
    \phantomsection
    \centerline{{#1}}
    \addcontentsline{toc}{section}{{#1}}
}
%----------------------------------------------------------------------%

%----Библиография: отступы и межстрочный интервал----------------------%
\makeatletter
\renewenvironment{thebibliography}[1]
    {\section*{\refname}
        \list{\@biblabel{\@arabic\c@enumiv}}
           {\settowidth\labelwidth{\@biblabel{#1}}
            \leftmargin\labelsep
            \itemindent 16.7mm
            \@openbib@code
            \usecounter{enumiv}
            \let\p@enumiv\@empty
            \renewcommand\theenumiv{\@arabic\c@enumiv}
        }
        \setlength{\itemsep}{0pt}
    }
\makeatother
%----------------------------------------------------------------------%
